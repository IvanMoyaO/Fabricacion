\begin{table}[H]\centering

\begin{tabular}{|l|r|r|r|}\hline
\rowcolor[HTML]{C0C0C0}\textbf{Línea de Unión} &\textbf{Tipo de Caña} &\textbf{Cabeza} &\textbf{Longitud \tablefootnote{Longitud sin Deformar} [mm]} \\\hline
Unión Sombrerete - Costilla &Caña maciza &Universal &8 \\\hline
Revestimiento borde de ataque – Costilla ($\times 3$) &Ciego y caña maciza &Universal &8 \\\hline
Zona plana revestimiento del borde de salida - Costillas ($\times 3$) &Ciego &Avellanada \tablefootnote{Zona con función aerodinámica, por tanto se ha de usar remachado avellanado. Se calcula usando cabezas de tipo universal para facilitar los cálculos. \label{av}}  &8 \\\hline
Zona curva revestimiento del borde de salida- Costillas ($\times 3$) &Ciego &Avellanada \footref{av}  &8 \\\hline
Revestimiento borde de ataque – Revestimiento extradós – Costillas ($\times 3$) &Caña maciza &Avellanada &9 \tablefootnote{Estos remaches unen ambos revestimientos, cuyo espesor total es de 3 [mm], luego se necesitan remaches con una longitud de 9 [mm] \label{9}} \\\hline
Revestimiento borde de ataque – Revestimiento intradós – Costillas ($\times 3$) &Ciego &Avellanada \footref{av} &9 \footref{9} \\\hline
Revestimiento borde de salida – Revestimiento intradós – Costillas ($\times 3$)&Ciego &Avellanada \footref{av} &9 \footref{9} \\\hline
Revestimiento extradós – Costilla ($\times 3$)&Caña maciza &Avellanada \footref{av} &8 \\\hline
Revestimiento intradós – Costilla ($\times 3$) &Caña maciza &Avellanada &8 \\\hline
Revestimiento borde de salida – Revestimiento extradós – Costillas ($\times 3$) &Caña maciza &Avellanada \footref{av} &9 \footref{9} \\\hline
Larguerillo delantero – Revestimiento intradós &Ciego &Avellanada \footref{av} &8 \\\hline
Larguerillo delantero – Revestimiento extradós &Caña maciza &Avellanada \footref{av} &8 \\\hline
Larguerillo trasero – Revestimiento borde de salida &Ciego &Avellanada \footref{av} &8 \\\hline
Tapa acceso - Costilla exterior ($\times 2$) &Caña maciza &Universal &8 \\ \hline
\end{tabular}
\end{table}