\begin{abstract}

El presente documento ha sido elaborado para el trabajo relacionado con las prácticas de Soldadura y Chapistería de Fabricación Aeroespacial (2º Cuatrimestre del Curso 23-24), por el grupo SJ11.

El trabajo versa sobre el proceso de fabricación y montaje de un perfil aerodinámico fabricado en acero y cuyo interior ha de ser accesible. Las cotas quedan recogidas en la Tabla \ref{tab:datos}. Los demás datos geométricos están en el Anexo.

Los Planos de Fabricación han sido elaborados en Catia. El Informe ha sido elaborado en LaTex, basándose en una plantilla elaborada por el Prof. Daniel del Río Velilla.

\begin{table}[!htb] \centering 
\begin{tabular}{|c|c|}\hline
    L1 &1000 \\\hline
    L2 &400 \\\hline
    L3 &260 \\\hline
    L4 &160 \\\hline
    \rowcolor[HTML]{ffcccb} L5 & 0 \\\hline
    \rowcolor[HTML]{ffcccb} L6 & 0 \\\hline
\end{tabular}
\quad
\begin{tabular}{|c|c|}\hline
    R1 &200 \\\hline
    R2 &960 \\\hline
    \rowcolor[HTML]{ffcccb} R3 &0 \\\hline
    \rowcolor[HTML]{ffcccb} R4 &0 \\\hline
    \rowcolor[HTML]{ffcccb} R5 & 0 \\\hline
    \rowcolor[HTML]{ffcccb} R6 & 0 \\\hline
\end{tabular}
\quad
\begin{tabular}{|c|c|}\hline
    $\Phi$1 &160 \\\hline
\end{tabular}


\caption{Cotas [mm] \label{tab:datos}}
\end{table}



\end{abstract}


